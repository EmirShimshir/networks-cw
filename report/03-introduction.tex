\chapter*{ВВЕДЕНИЕ}
\addcontentsline{toc}{chapter}{ВВЕДЕНИЕ}

Статический веб-сервер~---~сервер, который передает статические файлы в ответ на запросы от клиента. 
Статические файлы~---~файлы, содержимое которых не изменяется динамически на сервере.
Такой сервер предназначен для чтения файлов с диска и их отправки клиенту, обычно через протокол HTTP.

Целью курсовой работы является разработка сервера для отдачи статического содержимого с диска.
Архитектура сервера должна быть основана на предварительном создании процессов (prefork) совместно с epoll.

Для достижения поставленной цели необходимо решить следующие задачи:
\begin{enumerate}
	\item описать предметную область;
	\item спроектировать схему алгоритма работы сервера;
	\item выбрать средства реализации сервера;
	\item провести сравнение реализованного сервера с известными аналогами.
\end{enumerate} 
