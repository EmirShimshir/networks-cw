\chapter{Аналитический раздел} 

\section{Архитектура prefork}

Архитектура prefork~---~модель многозадачности, при которой для обработки входящих запросов заранее создается множество дочерних процессов процессов. 
В отличие от моделей с использованием потоков, prefork использует процессы, что исключает проблемы, связанные с синхронизацией потоков в рамках одного процесса.

Эта архитектура широко используется в веб-серверах, таких как Apache HTTP Server~\cite{apache_http_server}, где она обеспечивает изоляцию запросов и повышенную стабильность за счет независимости процессов друг от друга. 
Каждый процесс запускается заранее и ждет подключения клиента, что снижает накладные расходы на создание новых процессов при каждом запросе~\cite{prefork}. 

Основными компонентами архитектуры prefork являются:
\begin{enumerate}
	\item пул процессов;
	\item менеджер процессов.
\end{enumerate}

Менеджер процессов создает множество дочерних процессов (пулл процессов) и занимается мониторингом, для их восстановления при независимости.
\clearpage

\section{Epoll}

Epoll~---~это механизм ввода-вывода в Linux, предназначенный для эффективного управления большим количеством файловых дескрипторов. 
Он предоставляет интерфейс для мониторинга множества событий на различных файловых дескрипторах и позволяет эффективно реагировать на изменения их состояния~\cite{epoll}.

Epoll pеализует модель асинхронного ввода-вывода с использованием многозадачности на основе событий. 
Эта модель позволяет серверу эффективно отслеживать состояния множества файловых дескрипторов без блокировки потока выполнения.

Модель асинхронного ввода-вывода представлена на рисунке~\ref{img:iomodel}.
\includeimage
	{iomodel}
	{f}
	{H}
	{1\textwidth}
	{Модель асинхронного ввода-вывода}

