\chapter{Исследовательский раздел}

\section{Технические характеристики и описание исследования}

Технические характеристики устройства, на котором выполнялось тестирование:
\begin{enumerate}
	\item операционная система~---~Linux ubuntuserver 5.15.0-107-generic~\cite{linux_ubuntu}; 
	\item объем оперативной памяти~---~8 Гбайт;
	\item процессор~---~Intel Core I5, 4 ядера~\cite{intel}.
\end{enumerate}

Во время проведения экспериментов ноутбук находился в режиме работы от сети и был задействован исключительно встроенными приложениями и программой тестирования.

Для сравнения производительности был выбран веб-сервер \texttt{nginx}~\cite{nginx}, так как он считается одним из наиболее популярных и широко используемых решений.  
В качестве инструмента для создания нагрузки и измерения времени отклика применялся инструмент \texttt{ab} (Apache Benchmark)~\cite{apache_benchmark}.  
Основная задача исследования заключалась в сравнении производительности разработанного веб-сервера и известных аналогов на примере обработки запроса для получения файла размером 12.7 КБ при различном количестве запросов.

\clearpage
\section{Результаты исследования}

Результаты сравнения веб-серверов представлены в таблице~\ref{table1}.
\begin{table}[!ht]
	\centering
	\caption{Среднее время ответа на запрос для получения файла}
	\label{table1}
	\begin{tabularx}{\textwidth}{|X|X|X|}
		\hline
		Количество запросов & Среднее время ответа (разработанный веб-сервер), мс & Среднее время ответа (nginx), мс \\ \hline
		50 & 0.967 & 1.213 \\ \hline
		100 & 1.254 & 1.678 \\ \hline
		1000 & 1.345 & 2.845 \\ \hline
		5000 & 1.743 & 4.315 \\ \hline
		10000 & 2.077 & 5.421 \\ \hline
		50000 & 2.854 & 7.933 \\ \hline
		100000 & 3.014 & 9.102 \\ \hline
	\end{tabularx}
\end{table}

\section*{Вывод}

В исследовательском разделе было проведено сравнение реализованного статического веб-сервера и веб-сервера \texttt{nginx} по времени получения файла размером 12.7 КБ.
Из таблицы~\ref{table1} видно, что с увеличением количества запросов среднее время отклика для обоих серверов увеличивается, однако для разработанного веб-сервера это увеличение менее выражено, чем для \texttt{nginx}.
При сравнении двух серверов, можно заметить, что разработанный веб-сервер стабильно показывает меньшее время обработки, с разницей в среднем около 0.4 миллисекунд на меньших объемах нагрузки (до 1000 запросов), и эта разница увеличивается до 2–2.5 миллисекунд для большего количества запросов (более 1000). 
В целом, разработанный веб-сервер демонстрирует более высокую производительность по сравнению с \texttt{nginx}, особенно при высокой нагрузке, где разница в времени отклика может достигать 3 раз.
